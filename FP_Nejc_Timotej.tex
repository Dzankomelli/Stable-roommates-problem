\documentclass[12pt, a4paper]{article}
\usepackage[utf8]{inputenc}
\usepackage[T1]{fontenc}
\usepackage[slovene]{babel}
\usepackage{lmodern}
\usepackage{amsmath}
\usepackage{units}
\usepackage{eurosym}
\usepackage{amsfonts}
\usepackage{fancyhdr,amssymb,amsmath,amsthm,bbm,enumerate,mdwlist,url,multirow,hyperref,graphicx}
\usepackage{pdfpages}
\usepackage{comment}
\usepackage{breqn}
\usepackage{caption}
\usepackage{subcaption}
\usepackage{float}

\usepackage{enumerate}
\setlength{\parindent}{0mm}

\DeclareUnicodeCharacter{2212}{-}

\begin{document}
\begin{titlepage}
\begin{center}

\large
Univerza v Ljubljani\\
\normalsize
Fakulteta za matematiko in fiziko\\

\vspace{5 cm} 

\large
Finančni praktikum \\


\vspace{0.5cm}
\LARGE
\textbf{Stable roommate problem}

\vspace{0.5 cm}

\large
Timotej Giacomelli in Nejc Duščak\\


\vspace{1.5cm}
\normalsize
Mentorja: prof. dr. Sergio Cabello, asist. dr. Janoš Vidali
\vspace{3cm}


\vfill

\large Ljubljana, 2020

\end{center}
\end{titlepage}


\newpage

\tableofcontents
\vspace{22mm}

\newpage

\section{Uvod}
V projektu pri finančnem praktikumu bova obravnavala \textit{Stable roommate problem}.  Problem bova modelirala in poganjala ekspiremente v programskem jeziku Sage.\\

\textit{Stable roommate problem}, znan tudi kot kratica \textbf{SR}, je eden izmed \textit{stable matching} problemov, katere sta prvič predstavila David Gale in Lloyd Shapely. Problem je dobil ime zaradi svoje praktične uporabe - kako razporediti ljudi v dvoposteljne sobe, glede na njihove preference.\\

Problem je sestavljen iz $2n$ "udeležencev", kjer ima vsak udeleženec seznam preferenc s $2n - 1$ elementi, torej po eno vrednost za vsakega soudeleženca. Vsak udeleženec predstavljen točko v metričnem prostoru, njegov seznam pa so urejene dolžine do ostalih soudeležencev.\\

Ujemanje je množica $n$ disjunktnih parov udeležencev.
Za ujemanje $M$ je par $\{m_{1}, m_{1}' \} \notin M$ \textit{blocking pair}, če zadošča naslednjim pogojem:
\begin{itemize}
	\item  $\{m_{1}, m_{1}' \}, \{m_{2}, m_{2}' \} \in M$,
	\item $m_1$ preferira $m_2$ bolj kot $m_1'$,
	\item $m_2$ preferira $m_1$ bolj kot $m_2'$.
\end{itemize}
Oziroma če povemo z besedami, \textit{blockin pair} nastane, če se imata vsaj dva udeleženca, ki nista v paru, pri ujemanju raje, kot s svojim partnerjem.
Ujemanje $M$ je nestabilno, če zanj obstaja \textit{blocking pair}. Drugače je ujemanje $M$ stabilno.\\

Cilj SR je najti stabilno ujemanje ali pokazati, da nobeno ne obstaja. S časoma so uspeli razviti algoritem s časovno zahtevnostjo $O(n^2)$, ki bodisi najde stabilno ujemanje, bodisi ugotovi, da za dani primer ne obstaja nobeno stabilno ujemanje.\\

Stable roommate problem je v splošnem lahko uporabljen za ujemanje opazovanj in objektov pri nalogi razvrščanja. Na primer v življenjskem primeru iskanje primernega sostanovalca, so lahko le-ti predstavljeni po točkah v nekem prostoru lastnosti: koordinatna os prostora je lahko najprimernejši čas za spanje, želena raven urejenosti prostora, število zabav/piv na semester, itd.. Povsem logično je sklepati, da bo izbran udeleženec tisti, ki bo imel podobne lastnosti. \\

\pagebreak
Najin plan za naprej:
\begin{itemize}
	\item generirala bova $2n$ naključnih točk v kvadratu velikosti 1x1,
	\item izračunala bova razdalje med točkami,
	\item razdalje bodo predstavljale najine preference (manjša razdalja je večja preferenca), ki jih bova uredila po velikosti,
	\item napisala bova algoritem, ki bo izračunal ujemanje,
	\item analizirala bova, ali se seštevek razdalj med točkami v paru povečuje, ali zmanjšuje, ko povečujeva število točk ($n$).
\end{itemize}








\end{document}